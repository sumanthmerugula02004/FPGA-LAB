%\documentclass{article}
\documentclass[14pt]{article}
\usepackage{karnaugh-map}
\usepackage{circuitikz}
\usepackage[width=29.70cm, height=21.00cm, left=2.00cm, right=2.00cm, top=2cm, bottom=2cm]{geometry}
\title{EE5803 FPGA Lab Assignment 1}
\author{EE22MTECH02004  Sumanth Naidu Merugula}
\begin{document}
\maketitle
\section{Question}
[CBSE 2018 Q6 (d)] : Reduce the following Boolean Expression to its simplest form using K-map $$G(U,V,W,Z)=\sum(3,5,6,7,11,12,13,15)$$          
\section{Solution}
\subsection{Truth table}
The truth table for the given boolean expression $G$ is as follows
\begin{table}[h!]
\centering
\begin{tabular}{|c|c|c|c|c|} 
\hline
$U$ & $V$ & $W$  & $Z$ & $G(U,V,W,Z)$  \\ 
\hline
0 & 0 & 0 & 0 & 0     \\ 
\hline
0 & 0 & 0 & 1 & 0     \\ 
\hline
0 & 0 & 1 & 0 & 0     \\ 
\hline
0 & 0 & 1 & 1 & 1     \\ 
\hline
0 & 1 & 0 & 0 & 0     \\ 
\hline
0 & 1 & 0 & 1 & 1     \\ 
\hline
0 & 1 & 1 & 0 & 1     \\ 
\hline
0 & 1 & 1 & 1 & 1     \\ 
\hline
1 & 0 & 0 & 0 & 0     \\ 
\hline
1 & 0 & 0 & 1 & 0     \\ 
\hline
1 & 0 & 1 & 0 & 0     \\ 
\hline
1 & 0 & 1 & 1 & 1     \\ 
\hline
1 & 1 & 0 & 0 & 1     \\ 
\hline
1 & 1 & 0 & 1 & 1     \\ 
\hline
1 & 1 & 1 & 0 & 0     \\ 
\hline
1 & 1 & 1 & 1 & 1     \\ 
\hline
\end{tabular}
\caption{Truth table for Function G}
\end{table}

\subsection{K-map for simplification}
\begin{figure}
\begin{center}
\begin{karnaugh-map}[4][4][1][$WZ$][$UV$]
        \minterms{3,5,6,7,11,12,13,15}
        \maxterms{0,1,2,4,8,9,10,14}
        \implicant{7}{6}
        \implicant {12}{13}
        \implicant {3}{11}
        \implicant {5}{15}    
\end{karnaugh-map}
\end{center}
\caption{K-map for given boolean expression}
\label{kmap}
\end{figure}


Simplified expression from K-map can be written as $$G = WZ + VZ + UVW' + U'VW$$ 
Please refer to Figure 1. 

\subsection{NAND logic diagram}
To realize the above equation using NAND logic, the following steps are followed 
$$(G')' = ((WZ + VZ + UVW' + U'VW)')'$$
$$G = ((WZ)'(VZ)'(UVW')'(U'VW)')'$$
Please refer to Figure 2.
\begin{figure}
\begin{center}
\begin{circuitikz}
\draw
(0,0)node[nand port](mynand1){}
(0,2)node[nand port](mynand2){}
(0,4)node[nand port](mynand3){}
(0,6)node[nand port](mynand4){}
(0,8)node[nand port](mynand5){}
(2,1)node[nand port](mynand6){}
(2,4)node[nand port](mynand7){}
(4,1)node[nand port](mynand8){}
(4,3)node[nand port](mynand9){}
(6,1)node[nand port](mynand10){}
(8,2)node[nand port](mynand11){}
(8,7)node[nand port](mynand12){}
(10,2)node[nand port](mynand13){}
(10,7)node[nand port](mynand14){}
(12,4)node[nand port](mynand15){}


(mynand5.out)|-(mynand12.in 1)
(mynand4.out)|-(mynand12.in 2)
(mynand12.out)|-(mynand14.in 1)
(mynand12.out)|-(mynand14.in 2)
(mynand14.out)|-(mynand15.in 1)
(mynand1.out)|-(mynand6.in 1)
(mynand6.out)|-(mynand8.in 1)
(mynand6.out)|-(mynand8.in 2)
(mynand8.out)|-(mynand10.in 1)
(mynand10.out)|-(mynand11.in 2)
(mynand9.out)|-(mynand11.in 1)
(mynand11.out)|-(mynand13.in 1)
(mynand11.out)|-(mynand13.in 2)
(mynand13.out)|-(mynand15.in 2)
(mynand3.out)|-(mynand7.in 1)
(mynand3.out)|-(mynand7.in 2)
(mynand2.out)|-(mynand9.in 2)
(mynand7.out)|-(mynand9.in 1)


(mynand1.in 1) -- (mynand1.in 2)
(mynand1.in 1 |- mynand1.out) to [short,*-] (-2,0)
(mynand2.in 1) -- (mynand2.in 2)
(mynand2.in 1 |- mynand2.out) to [short,*-] (-2,2)

(mynand5.in 1) node      [anchor=east]           {Z}
(mynand5.in 2) node      [anchor=east]           {W}
(mynand4.in 1) node      [anchor=east]           {Z}
(mynand4.in 2) node      [anchor=east]           {V}
(mynand3.in 1) node      [anchor=east]           {U}
(mynand3.in 2) node      [anchor=east]           {V}
(mynand2.in 1) node      [anchor=east]           {W}
(mynand1.in 1) node      [anchor=east]           {U}
(mynand6.in 2) node      [anchor=east]           {V}
(mynand10.in 2) node     [anchor=east]           {W}
(mynand15.out) node      [anchor=west]           {G};
\end{circuitikz}
\end{center}
\caption{Logic circuit using NAND gate}
\label{ckt1}
\end{figure}

\end{document}

