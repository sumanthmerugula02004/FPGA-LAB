\documentclass{article}
\usepackage{karnaugh-map}
\title{EE5803 FPGA Lab Assignment 1}
\author{EE22MTECH02004  Sumanth Naidu Merugula}
\begin{document}
\maketitle
\section{Question}
[CBSE 2018 Q6 (d)] : Reduce the following Boolean Expression to its simplest form using K-map $$G(U,V,W,Z)=\sum(3,5,6,7,11,12,13,15)$$          
\section{Solution}
Simplification using K-map:

\begin{karnaugh-map}[4][4][1][$WZ$][$UV$]
        \minterms{3,5,6,7,11,12,13,15}
        \maxterms{0,1,2,4,8,9,10,14}
        \implicant{7}{6}
        \implicant {12}{13}
        \implicant {3}{11}
        \implicant {5}{15}     
\end{karnaugh-map}

Simplified expression from above map can be written as $$G = WZ + VZ + UVW' + U'VW$$ 
NAND realization : To realize the above equation using NAND logic, the following steps are followed 
$$(G')' = ((WZ + VZ + UVW' + U'VW)')'$$
$$G = ((WZ)'(VZ)'(UVW')'(U'VW)')'$$
\end{document}

